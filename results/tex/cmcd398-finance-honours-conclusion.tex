In this paper, I document the outcomes from assessing the accuracy and relative performance
in predicting hedge portfolio excess returns between deep neural networks using a 
maximisation of a hedge portfolio excess return optimisation function, and conventional mean squared error loss minimisation functions.
I document the mathematical derivations for a non-convex optimisation function incorporating a monotonic ranking function
to weight stocks proportionally by their contributions to aggregate excess returns, supported by mathematical theory surrounding ordinary least squares (OLS) regressions.
I document the infrastructure and extensive programming to conduct analysis.
Next, I record the maximisation function makes well generalised out-of-sample predictions given the in-sample training processes,
but all models performed inconsistently when evaluating and interpreting performance measures from the lens of economic significance.

Gu et al. (\citeyear{eapvml})'s pivotal paper frames the feasibility of machine learning for empirical asset pricing via machine learning in attempts to advance the field.
Data science in finance continues to evolve in academic and industry-related uses.
Most machine learning applications in equity return predictions, and finance in general, use a traditional loss functions.
A handful of researchers explore custom loss minimisation functions in machine learning applications, but none from the perspective of maximising hedge portfolio excess returns as far as can be told.

Furthermore, I hypothesised it was feasible to train a neural network using a hedge portfolio excess return optimisation strategy, 
but would not outperform traditional loss minimisation functions given neural network architecture and the mathematics surrounding loss functions.
Both hypotheses prove to be true with the given dataset.

The analysis and outcome show maximisation strategies are feasible from a technical perspective. This is a partial, novel contribution to the literature.
Another minor contribution is the validation of using of factor portfolio dataset for this form of analysis.
However, the limitations relating to resources, data revisions, optimisation functions, neural network architecture, and simulations, render the research
question inconclusive at this stage. Further analysis will continue to work towards exploring and resolving these issues in the next chapter.