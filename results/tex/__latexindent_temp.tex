Factor-based models feature prominently in finance literature, in forms of cross-sectional analysis on expected returns as a function of stock level characters.
Another form is the surveillance of return time-series.
However, limitations exist for both traditional methods in consideration a large set of characteristics or periods.
Proliferating data, advancement in computing, and accessibility to technology led to a wide adoption of artificial intelligence and machine learning.
The pivotal publication by Gu et al. (\citeyear{eapvml}) illustrates the improvement to empirical asset pricing via machine learning.
However, interpretability issues persist given the nature, opacity, and complexity of underlying algorithms and associations.
Fortunately, acceptance continues to improve.
Data science in finance continues to evolve in academic and industry-related uses.
Most machine learning applications in equity return predictions, and finance in general, use a traditional loss function.

Our main motivation roots itself in economic significance.
Investors' interests lie in maximising the hedge portfolio excess returns from their hedging strategies.
Machine learning algorithms capture non-linear associations between factors not feasible by other asset pricing method.
Non-linear associations may exist in constructing hedge portfolios absent when predicting excess returns from individual equities.
Subsequently, we ask can neural networks configured to maximise hedge portfolios outperform standard loss minimisation configurations,
when predicting one month lead excess returns using a long-short zero cost investment strategy?

Hou et al. (\citeyear{hou2020replicating}) use an extensive data library to assess 452 anomalies across anomalies literature.
Jensen et al., \citeyear{jensen2021there} use the above dataset to explore hierarchical bayesian models of alphas emphasising the joint behaviours of factors, 
and provide an alternative multiple testing adjustment, more powerful than common methods.
The complete global dataset has 406 characteristics, a superset of the original 153 in Jensen et al., with 2,739,928 firm-year observations, from January 1st 1961 to December 31st 2020.
One month lead excess returns is the target variable for prediction, informing hedge portfolios construction to assess the relative performance between loss functions.
The exhaustive nature and accessibility of the global dataset makes it well-suited for exploring maximising monthly hedge portfolio excess returns in deep neural-networks.
Neural networks demand the partitioning of the dataset into training, validation, and testing subsets.
The initial training, testing, and validation sets consist of \textbf{1031516}, \textbf{706908}, and \textbf{1001504} global equity firm-year observations across 406 features, respectively.
The division of subsets is chronological with firm-year observations [1961-1990), [1990-2000), [2000-2020] for training, validation, and testing, respectively.
The training, testing, and validation sets are reduced to consist of \textbf{532218}, \textbf{294581}, and \textbf{531461} global equity firm-year observations across 160 features after revisions, respectively.

Cloud-centric computational infrastructure performs data processing and analysis.
Deep neural networks require tensors as inputs for fitting, training, and evaluating data.
Normalisation and encoding processes transform the aforementioned dataset into a tensor format.
Artificial Neural Nets (ANN) frequently outperform other machine learning algorithms on large and complex problems.
The architecture of the network is derivative of intended use.
The neural network configuration for analysis has three hidden dense layers with ReLU activation functions, an output layer with a linear activation function, a dropour layer to prevent overfitting, and a stochastic gradient descent optimiser.
This network has the optimal layer density as per Gu et al. (\citeyear{eapvml}).
Loss functions map an event or variable set, onto a real number, intuitively representing some loss, associated with the event e.g., difference between predicted and realised excess returns.
The relative performance between conventional minimisation loss functions, and maximisation of hedge portfolio excess returns, relies on comparing derivatives to these objectives.
The analysis tests three loss functions. First, a mean squared error measure optimised for Tensorflow. Second, a custom squared error function to ensure automatic differentiation functionalities in Tensorflow.
Last, a non-convex function seeking to maximise hedge portfolio returns with hedge portfolio weights determined by a monotonic ranking function mapping.
The selected mapping weights individual equities by the proportion of their contributions to aggregate returns of all equities in a given month, considering all equities in the portfolio.
Performance metrics inform comparisons between loss functions, calculated by:

First, the trained models predict one month lead excess returns for each instance (firm-year observation) in the testing dataset.
Second, standard monthly sorts of predicted one month lead excess returns form standard tercile hedge portfolios, using a long-short zero cost investment, per month.
Third, use hedge portfolios to calculate the hedge portfolio mean, sharpe ratio, and treynor ratio across all months.
Last, Ordinary least squares regressions, incorporating Newey-West estimators and six month lags, calculating coefficients between realised and predicted individual stock returns, and alpha on from the Capital Asset Pricing Model, and Fama-French Factors.

Both the $MSE$($\hat{y}$,y) and $MSE_{Custom}$($\hat{y}$,y) final training losses are near identical at 0.011 and 0.010, respectively.
The pattern for validation losses is similar at 0.017 and 0.014 for $MSE$($\hat{y}$,y) and $MSE_{Custom}$($\hat{y}$,y), respectively.
The Hedge Portfolio loss function has a minimum training and validation loss of 0.01 and 0.001 before running for a further three epochs before early termination, respectively.
The learning curves fit well.
The F Statistic on all tests in regressing actual individual excess returns on prediction is highly statistically significant at the one percent level, 
verifying realised returns are derivative of predictions.
The training losses, validation losses, learning curves, and regressed realised returns on predictions show evidence of appropriate fitting.
This confirms the first hypothesis in our motivations that maximising hedge portfolio returns is feasible as an optimisation function in a neural network.

The hedge portfolio means for $MSE$($\hat{y}$,y), $MSE_{Custom}$($\hat{y}$,y), and Hedge Portfolio ($\hat{y}$) loss functions are 0.0424, 0.0764 and, -0.0168 per month, respectively.
These raise concern given they exceed observations from prior literature. However, these may be caused by data reductions further described in the main text.
The sharpe ratios for both MSE variants are feasible at ~1.15 to 1.45 for $MSE$($\hat{y}$,y), and $MSE_{Custom}$($\hat{y}$,y).
Ultimately, -0.016803 and -0.714680 for hedge portfolio mean and sharpe ratio shows the hedge portfolio maximisation strategy does not outperform traditional loss minimisation regardless of accuracy.
$MSE$($\hat{y}$,y) (table \ref{mse-tf-apm}), $MSE_{Custom}$($\hat{y}$,y) (\ref{mse-apm-cmse}), and Hedge Portfolio ($\hat{y}$) (\ref{hp-apm-hp}) neural networks generate portfolios which align with their hedge portfolio means, across all factor models, statistical significant at the 1\% level.
The MSE variants only have highly statistical significant negative market premia factors, and the Hedge portfolio have statistical significant negative RMW and SMB in FF4 and FF5 (5\% level).
The sample period, incomplete dataset, shorting positions in the hedge portfolios, and capped weighting methodology may be driving these results.
Regardless, the results are inconclusive in assessing relative performance.
However, if they were, the hedge portfolio maximisation strategy would not outperform traditional loss minimisation techniques.

The analysis and outcome show maximisation strategies are feasible from a technical perspective. This is a partial, novel contribution to the literature.
Another minor contribution is the validation of using of factor portfolio dataset for this form of analysis.
However, the limitation relating to resources, data revisions, optimisation functions, neural network architecture, and simulations render the research
question inconclusive at this stage. Further analysis will continue to work towards exploring and resolving these issues in the next chapter.
The subsequent sections appear as follows: Literature (\ref{LR}), Motivation (\ref{motivations}), Methodology (\ref{methodlogy}), 
Results (\ref{results}), Research Implications (\ref{research-implications}), Conclusion (\ref{conclusion}), and Appendix (\ref{appendix}).\footnote{The appendix include documentation on the programming necessary to complete analysis (50+ custom functions and classes), accessible on \href{https://github.com/CMCD1996/finance-honours}{Github}}

